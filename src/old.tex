
\chapter*{Acknowledgments}
Optionally, an acknowledgements section can be added to express gratitude to those who contributed to the work. Appreciation can be extended to the advisors, faculty members, peers, family, and friends for their support. It is advisable to maintain conciseness in this section, and it is important to note that inclusion is at the author's discretion.

\chapter*{Abstract}
An abstract is a concise summary of the key points of your manuscript. It is usually placed at the beginning of the document and provides a brief overview of the study/research. The goal is to give the potential readers a quick understanding of the content so they can determine if the full document is relevant to their interests. In general, technical details should not be described in the abstract. Finally, the abstract should mention the conclusion, giving a perspective on what was addressed and what was solved by the work described in the document.

\begin{itemize}
    \item What is the work about?
    \item What has been done in general terms?
    \item What is the impact of the work?
    \item What results have been achieved / what problems have been solved?
\end{itemize}

\clearpage

\chapter{State of the Art}
This chapter contains an analysis of the previous works that aim to solve the same (or very similar) problem(s). The chapter should focus on the most recent and effective techniques and technologies, especially considering the approaches that are in use. Studying the state of the art is supposed to be one of the first activities to be completed during the thesis/stage period, since it allows the student to better understand what has been done so far, what should be better investigated, and what the baseline tools and approaches are to consider.
To understand how to organize this chapter, it is usually useful to read review papers (papers that collect, analyze and compare methods proposed by other papers).
If the thesis/final report is related to a software development project, this chapter should critically describe the technologies and main architectural choices of previous related works.
Do not forget that this section describes the state of the art about the specific problem considered, and it is needed to just properly position the work.

\begin{itemize}
    \item What are the best methods currently available for solving the addressed problem?
    \item How do they work?
    \item What are weaknesses and limitations in existing approaches?
    \item How does the work described in the document relate to the existing body of work?
\end{itemize}

\chapter{Proposed Approach}
This is the methodology section of your manuscript, outlining the methods and procedures that were planned to use to address the problems and questions. At the beginning, it should briefly reiterate the problem or research question, reminding the readers of the context of the work. 
If the work consists of a new approach, the section should motivate how the method has been designed, giving an overview of the method and then presenting in detail its individual steps, also using examples. 
If the work consists of an empirical study, this section should motivate and explain the overall design of the study, and then describe in detail the steps of the empirical methodology. The goal here is to provide a clear and comprehensive plan for the research/work. The student is expected to be transparent about the methods, thoughtful in addressing potential challenges, and (if any) ethical considerations. The goal is to convince readers that the chosen methods are appropriate and will effectively address the question or objectives.

\begin{itemize}
    \item What is the methodology? Why is the methodology right?
    \item How are experiments and procedures incorporated in the work?
    \item How does the proposed method work?
    \item How can the student ensure that the chosen methods will effectively address the problem?
\end{itemize}

\chapter{Analysis and results}
This is one of the most important sections of the manuscript since it allows the student to critically analyze the method and its results . If a new approach was proposed, it should be compared to previous approaches/state-of-the-art methods. It is essential to emphasize the pros and cons of each approach. Depending on the nature of the work, results might look very different. For example, if it involves a ML model, it can be tested on multiple datasets and with multiple performance metrics. If a new method was introduced in a company, the section might point out the resulting improvements in productivity. The student should highlight the strengths and weaknesses of the described methods, and the problems that remain yet to be solved.

\begin{itemize}
    \item What is good/bad about the proposed/analyzed method(s)?
    \item In what aspects is it better/worse than other existing methods in the state of the art?
    \item What is still yet to be solved? Why is it difficult? Are there some limitations?
\end{itemize}

\chapter{Conclusions}
The conclusion chapter serves as a summary of the key findings and results of the work, providing some closing statements on its overall significance. It is useful to recap the main results in a concise manner, and discuss how they align with the objectives stated in the introduction. It is also necessary to discuss the theoretical or practical implications of the results obtained and how they contribute to the existing body of knowledge or results that may be applied in a real-world scenario. In conclusion, some future developments of the work should be mentioned, with suggestions about what could be further improved and explored.

\begin{itemize}
    \item What are the main findings and results of the work?
    \item What is the relevance of the results, and how might they be applied in real-world scenarios?
    \item What limitations or challenges were encountered during the work?
    \item How does the manuscript provide closure to the reader and emphasize the overall significance?
    \item What are some future developments of the work done so far?
\end{itemize}
