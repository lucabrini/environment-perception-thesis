\chapter{Mappe semantiche}

La percezione dell'ambiente circostante è una delle attività più importanti per un robot, soprattutto nell'ambito del \textbf{Mission Planning}. La capacità di riconoscere gli oggetti e di calcolarne la posizione è fondamentale per poterci interagire. Inoltre, è essenziale potersi localizzare nella mappa, sia in modo geometrico che topologico, in modo da poter pianificare anche eventuali movimenti verso gli oggetti desiderati che si trovano in punti non raggiungibili al momento dal robot. In questo capitolo definiremo il significato di \textbf{Mappa Semantica}, la sua creazione nonché aggiornamento e illustreremo le ragioni che portano questa base di conoscenza ad essere efficace per il mission planning tramite Large Language Models.

\section{Stato dell'Arte}



\section{Approccio proposto}


\subsection{Generazione del Grafo di Scena}
\paragraph{Object coordinates}
\paragraph{World reference Object coordinates}

\subsection{Aggiormento del Grafo di Scena}
\paragraph{Proiezione del Camera Frustum}
\paragraph{Check if object is in camera frustum}


\section{Analisi e Risultati}

\subsection{Errore della posizione degli oggetti}
\subsection{Punti di forza e svantaggi}

\section{Conclusioni}

