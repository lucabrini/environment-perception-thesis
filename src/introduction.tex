
Negli ultimi anni il campo della robotica ha vissuto un significativo incremento di applicazioni e innovazioni. Lo sviluppo di nuove tecnologie e la disponibilità di nuovi strumenti hanno reso possibile la creazione di robot in grado di svolgere compiti sempre più complessi. La \textbf{pianificazione automatica delle missioni} è sempre stata una delle attività di sviluppo in questo campo più affascinanti, pur essendo una delle più tediosa. Con l'avvento di ChatGPT e modelli simili, si è iniziato a pensare di integrare i \textbf{Large Language Models}, come alternativa ai classici planner, all'interno del sistema robot, con l'obiettivo di pianificare missioni autonome sulla base della descrizione in linguaggio naturale di ciò che si vuole far eseguire al robot.

La percezione dell'ambiente circostante è dunque una delle attività più importanti per un robot, soprattutto nell'ambito del \textbf{Mission Planning}. La capacità di riconoscere gli oggetti e di calcolarne la posizione è fondamentale per poterci interagire. Inoltre, è essenziale potersi localizzare nella mappa, sia in modo geometrico che topologico, in modo da poter pianificare anche eventuali movimenti verso gli oggetti desiderati che si trovano in punti non raggiungibili al momento dal robot.

In questo documento definiremo il significato di \textbf{Mappa Semantica}, le ragioni alla base della sua esistenza, la struttura e come viene utilizzata per pianificare le missioni del robot. Successivamente entreremo nel dettaglio del \textbf{Grafo di Scena}, come viene generato e tenuto aggiornato con i cambiamenti dell'ambiente. Infine analizzeremo il \textbf{Riconoscimento delle Stanze} a partire dalla mappa SLAM generata attraverso i sensori LiDaR del Robot, essenziale per suddividere l'insieme degli oggetti nelle loro stanze e gestire le missioni che necessitano lo spostamento in altre stanze.