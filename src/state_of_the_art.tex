Il mapping semantico e la ricostruzione dell'ambiente sono aspetti, fondamentali per il planning di robot autonomi, che hanno ricevuto molta attenzione per diverso tempo. Sin dai primi lavori nel campo si è cercato di ricostruire l'ambiente in cui il robot opera definendo dei link tra rappresentazione spaziale e semantica dell'ambiente. \\
Per esempio, \cite{galindo} propone un sistema multi-gerarchico di conoscenza suddiviso in gerarchia spaziale, %strettamente legata ai sensori di tipo Lidar o Sonar e
utilizzata per la pianificazione e esecuzione di task e in gerarchia concettuale che modella la conoscenza semantica. Queste sono in relazione tra loro tramite una serie di \textit{anchor} che collegano nodi \textit{Thing} delle due gerarchie. Per quanto riguarda la costruzione del modello semantico, la label associata agli oggetti è assegnata dal sistema di visione artificiale mentre il tipo di stanza é inferito grazie a quali oggetti essa contiene.\\
L'approccio proposto da \cite{theobald_godot} integra il feedback da parte dell'utente nei \textit{beliefes} del sistema con l'obiettivo di ridurre l'incertezza riguardo le informazioni semantiche associate a un ambiente. La mappa è divisa in tre layer di rappresentazione: il layer geometrico, il layer topologico e il layer semantico. Il layer geometrico consiste nella mappa probabilistica di occupazione, il layer topologico è automaticamente costruito dalla mappa geometrica suddividendo lo spazio libero in regioni utilizzando l'algoritmo di Voronoi \cite{thrun} e il layer semantico estende quello topologico allegando label simboliche a gruppi di regioni.\\
Infine, con la crescente popolarità di modelli LLM e l'introduzione di GPT-4 Vision si è pensato di utilizzare i Large Language Models per l'identificazioni di \textit{regions of interest} all'interno di una immagine, come proposto da \cite{RunjiaTan}.\\\\
Il lavoro di questa tesi mira ad approfondire queste tematiche, unendo il meglio degli approcci sopra citati e cercando di migliorarne le lacune. Per quando riguarda la costruzione della mappa semantica, il riconoscimento degli oggetti è completamente autonomo, senza bisogno dell'intervento dell'umano. L'utilizzo di una variante dell'algoritmo di Voronoi permette di riconoscere le stanze all'interno della mappa geometrica e solo successivamente sarà l'utente a definirne il nome. Infine, il tutto viene implementato senza l'utilizzo di LLM che rallenterebbero il tutto rendendo l'applicazione non adatta al contesto real time. Di conseguenza, così come per gli essere umani, il sistema permette integrare continuamente nuove informazioni riguardo l'ambiente, senza necessità di doversi fermare e aspettare i risultati di determinati processi.