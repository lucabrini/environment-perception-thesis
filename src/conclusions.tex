Questo lavoro di tesi ha avuto come obiettivo principale la realizzazione di un sistema per percepire l'ambiente e costruire la conoscenza acquisita frame dopo frame. Il sistema è stato progettato per RoBee, robot umanoide cognitivo di Oversonic Robotics. \\\\
Come illustrato nei capitoli precedenti, il modello di PSGTr consente di acquisire informazioni sugli oggetti, e le relazione tra loro, presenti nell'ambiente in modo efficiente, caratteristica essenziale per un'applicazione real time di questo tipo. L'integrazione delle nuove informazioni sfrutta operazioni di geometria lineare, che si traducono in una serie di moltiplicazioni e addizioni molto veloci da eseguire per un calcolatore. Tuttavia, un miglioramento in questo senso si potrebbe ottenere utilizzando altri modelli che, sfruttando l'object tracking, consentono di ridurre il numero di operazioni necessarie per aggiornare la mappa semantica. Inoltre si potrebbe pensare di utilizzare un database a grafo, al posto dell'attuale MongoDB, per memorizzare la mappa semantica e sfruttare così le capacità di query offerte da un database di questo tipo. Un ulteriore miglioramento potrebbe essere quello di eseguire un fine tuning del modello PSGTr in base al contesto di utilizzo finale di RoBee.\\\\
Il sistema di riconoscimento delle stanze, basato sull'Algoritmo di Voronoi, consente di estrapolare a partire da una grid map di occupazione il grafo delle stanze della mappa semantica. I risultati ottenuti sono stati abbastanza soddisfacenti, per le varie mappe a disposizione di Oversonic. Tuttavia, non mancano gli errori che, grazie allo strumento di modifica delle stanze, possono essere innanzitutto corretti e in futuro si potranno utilizzare le informazioni ottenute per addestrare un modello ad-hoc per il riconoscimento delle stanze. \\\\
