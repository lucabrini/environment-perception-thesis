Questo lavoro di tesi ha avuto come obiettivo principale la realizzazione di un sistema per percepire l'ambiente e costruire la conoscenza acquisita frame dopo frame. Il sistema è stato progettato per RoBee, robot umanoide cognitivo di Oversonic Robotics, in particolare per supportare il sistema di Mission Planning tramite LLM data la richiesta in linguaggio naturale. \\
\\
Come illustrato nei capitoli precedenti, il modello di PSGTr consente di acquisire informazioni sugli oggetti, e le relazioni tra loro, presenti nell'ambiente in modo efficiente, caratteristica essenziale per un'\textbf{applicazione real time} di questo tipo. L'integrazione delle nuove informazioni sfrutta operazioni di geometria lineare, che si traducono in una serie di moltiplicazioni e addizioni molto veloci da eseguire per un calcolatore, sia su GPU che CPU. Tuttavia, un miglioramento in questo senso si potrebbe ottenere utilizzando altri modelli che, sfruttando l'\textbf{object tracking}, consentono di ridurre il numero di operazioni necessarie per aggiornare la mappa semantica e tenere traccia degli oggetti nel tempo. Inoltre si potrebbe pensare di utilizzare un \textbf{database a grafo}, al posto dell'attuale MongoDB, per memorizzare la mappa semantica e sfruttare così le capacità di query offerte da un database di questo tipo. Un ulteriore miglioramento potrebbe essere quello di eseguire un \textbf{fine tuning} del modello PSGTr in base al contesto di utilizzo finale di RoBee.\\
Per le basse performance di segmentazione panoptica di PSGTr, nonostante ci siano meccanismi per limitarne le conseguenze e le classi che presentano maggiori problemi siano quelle nella blacklist di filtraggio (come \textit{wall, floor, ceiling, etc.}), si potrebbero utilizzare \textbf{modelli separati} per la generazione del grafo di scena (dove PSGTr è già Stato dell'Arte ) e per la segmentazione panoptica.\\
\\
Il sistema di riconoscimento delle stanze, basato sull'Algoritmo di Voronoi, il cui obiettivo è quello di estrarre il grafo delle stanze della mappa semantica data una grid map di occupazione, consente di aggiornare solo la stanza attuale del robot, efficientando ulteriormente il processo. I risultati di riconoscimento ottenuti sono stati abbastanza soddisfacenti, per le varie mappe a disposizione di Oversonic. Tuttavia, non mancano gli errori che, grazie allo strumento di modifica delle stanze, possono essere corretti. Le \textbf{annotazioni} finali delle stanze, in futuro, si potranno utilizzare per \textbf{addestrare un modello} ad-hoc per il riconoscimento delle stanze.
