\newglossaryentry{backbone}{
  name=backbone,
  description={
      Architettura di rete neurale pre-addestrata che funge da base per ulteriori sviluppi e adattamenti specifici di una particolare applicazione. Grazie al transfer learning è possibile costruire architetture per task complessi sopra a questo modello. Utilizzata molto spesso nella visione artificiale come supporto
    }
}

\newglossaryentry{point_cloud}{
  name={point cloud},
  description={ Raccolta di dati che rappresenta oggetti o superfici tridimensionali. Ogni punto nella nuvola ha coordinate (x, y, z) che ne definiscono la posizione nello spazio. A volte, ai punti sono associati anche altri attributi, come colore o intensità
    }
}

\newglossaryentry{camera_system}{
  name={sistema camera},
  description={Sistema di coordinate che prevede la camera nell'origine, solitamente al centro. Tutte le coordinate espresse in questo sistema hanno quindi come riferimento la posizione della camera}
}

\newglossaryentry{pixel_system}{
  name={sistema pixel},
  description={Sistema di coordinate intere 2D con l'origine in alto a sinistra}
}

\newglossaryentry{transformation_matrix}{
  name={matrice di trasformazione},
  description={Matrice che viene utilizzata per descrivere come un oggetto o un insieme di punti nello spazio viene modificato da una trasformazione geometrica}
}

\newglossaryentry{homogeneous_transformation_matrix}{
  name={matrice di trasformazione affine},
  description={Matrice di trasformazione 4x4 che combina rotazione traslazione.
      \[
        \begin{bmatrix}
          R_{11} & R_{12} & R_{13} & T_x \\
          R_{21} & R_{22} & R_{23} & T_y \\
          R_{31} & R_{32} & R_{33} & T_z \\
          0      & 0      & 0      & 1
        \end{bmatrix}
      \]
      Dove:
      \begin{itemize}
        \item $R_{11}, R_{12}, R_{13}, ..., R_{33}$ sono gli elementi della matrice di rotazione 3x3.
        \item  $T_x, T_y, T_z$ sono le componenti del vettore di traslazione.
      \end{itemize}
      Attraverso una sola moltiplicazione di matrici è possibile applicare sia la rotazione che la traslazione a un punto $(x,y,z)$
      \[
        \begin{bmatrix}
          x' \\
          y' \\
          z' \\
          1
        \end{bmatrix}
        =
        \begin{bmatrix}
          R_{11} & R_{12} & R_{13} & T_x \\
          R_{21} & R_{22} & R_{23} & T_y \\
          R_{31} & R_{32} & R_{33} & T_z \\
          0      & 0      & 0      & 1
        \end{bmatrix}
        *
        \begin{bmatrix}
          x \\
          y \\
          z \\
          1
        \end{bmatrix}
      \]
      Il risultato $(x', y', z')$ rappresenta le nuove coordinate del punto dopo la trasformazione
    }
}

\newglossaryentry{tf_tree}{
  name={albero delle TF},
  description={Struttura ad albero che descrive le relazioni spaziali tra i diversi componenti di un robot. Ogni nodo dell'albero rappresenta un sistema di coordinate, e ogni ramo rappresenta una trasformazione (rotazione e traslazione) che collega un sistema di coordinate al suo sistema padre}
}

\newglossaryentry{kubeedge}{
  name={KubeEdge},
  description={Progetto open source che estende Kubernetes da un cluster cloud a un cluster edge, permettendo di eseguire i pod su dispositivi edge come robot, macchine industriali e così via}
}

\newglossaryentry{panopticquality}{
  name={Panoptic Quality (PQ)},
  description={La Panoptic Quality (PQ) è una metrica utilizzata per valutare la performance degli algoritmi di segmentazione panottica, che combinano segmentazione semantica e delle istanze. PQ misura la qualità della segmentazione e del riconoscimento in un'unica metrica, considerando la sovrapposizione tra le previsioni e i segmenti di riferimento (IoU), i veri positivi (TP), i falsi positivi (FP) e i falsi negativi (FN). La formula di calcolo della PQ è:
      \[
        \text{PQ} = \frac{\sum_{(p, g) \in TP} \text{IoU}(p, g)}{|TP| + \frac{1}{2}|FP| + \frac{1}{2}|FN|}
      \]
      Un valore più alto di PQ indica una migliore performance dell'algoritmo di segmentazione.}
}
